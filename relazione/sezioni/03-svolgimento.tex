%\clearpage
\section{Sviluppo}
Prima di iniziare lo sviluppo del programma ho avuto la necessità di approfondire lo studio del formato \acronimo{bam/sam} per comprendere come sfruttare al meglio la struttura del file stesso e delle librerie che aiutano l'estrazione dei dati: ciò è stato necessario, in quanto tale formato di file presenta molti campi per ogni read e solo alcuni di essi sono risultati essere davvero utili agli scopi del progetto.

Ho riservato inoltre una parte del tempo allo studio del linguaggio di programmazione scelto (\emph{Python}), in quanto non lo avevo mai utilizzato prima, e agli strumenti da utilizzare durante il progetto in modo da poter sfruttare al meglio le loro caratteristiche peculiari.

\subsection{Studio preliminare}
La prima fase dello sviluppo ha riguardato lo studio diretto dei file \acronimo{bam/sam}: tale attività mi è stata necessaria per capire quali dati erano necessari e qual era il modo migliore per estrarli.

È stato fondamentale infatti trovare degli algoritmi che esaminassero le read in modo efficiente, in quanto la notevole quantità delle stesse ha un grande impatto nei tempi di esecuzione del programma.

Inizialmente avevo scelto di utilizzare i due file \acronimo{pass} forniti in modo parallelo, mantenendo quindi i file separati e confrontando una read alla volta per ogni file; questo approccio mi ha consentito di trovare \emph{single read} e \emph{unique read} in modo semplice ma rendeva inutilmente complicato trovare le \emph{multiple read}.

Ho optato quindi per l'unione dei due file in un unico file, ordinato per nome delle read in modo da semplificare la ricerca delle informazioni.

\subsection{Script \emph{Python}}
Il primo passo è stato quello di individuare \emph{single}, \emph{unique} e \emph{multiple read} per poi poter proseguire con l'analisi e produrre i file \emph{wiggle} e \emph{gnuplot} da visualizzare.

Una volta suddivise le read per tipologia, vengono creati i relativi file \acronimo{bam} da utilizzare, in seguito, per calcolare le rispettive \emph{sequence coverage}, la \emph{physical coverage} e le \emph{insert length} per i \emph{mate pair}.

Lo script, chiamato \script{reseq\_new}, è quindi composto da quattro funzioni che si occupano di svolgere diversi calcoli:

\begin{description}
\item[\textsc{esamina\_bam}]: funzione che si occupa di contare le occorrenze di ogni read, sfruttando il campo \campo{qname} e costruisce tre liste diverse a seconda che vengano trovate \emph{single read}, \emph{unique read} o \emph{multiple read}.

Una volta costruite le liste, vengono creati i rispettivi file \acronimo{bam} e salvati su disco per poter essere utilizzati in seguito.

\item[\textsc{esamina\_multiple}]: funzione che si occupa di esaminare le \emph{multiple read} trovate e di suddividerle ulteriormente cercando di trovare ulteriori \emph{mate pair} validi.

Questa funzione cerca di sfruttare alcune caratteristiche delle read per trovare i giusti accoppiamenti: esiste però una grande variabilità di condizioni che permettono di scegliere quali read accoppiare ed è quindi complicato elaborare un algoritmo che svolga questo compito al meglio.

\item[\textsc{calcola\_insert\_length}]: funzione che calcola la lunghezza degli inserti, e quindi delle sole \emph{unique read}, scartando quelli con lunghezza molto superiore alla loro media ($20000bp$).

Viene inoltre calcolata la \emph{physical coverage} dei \emph{mate pair}: i campi relativi al file \acronimo{bam} che vengono usati sono \campo{pos} e \campo{seq} sfruttando i metodi offerti da \emph{pysam}.

Durante il calcolo vengono create diverse liste che verranno usate per scrivere i dati su file \emph{wiggle} e \emph{gnuplot}; tali file saranno poi usati per creare dei grafici che mostrano i risultati di questa elaborazione.

Al termine del calcolo vengono fornite la media e la deviazione standard delle lunghezze.

\item[\textsc{calcola\_coverage}]: funzione che calcola la \emph{sequence coverage} delle read, sfruttando i metodi offerti dal modulo \emph{pysam}, e scrive i file \emph{wiggle} da utilizzare per la visualizzazione dei risultati.
\end{description}

Sono stati sviluppati inoltre due moduli ausiliari che si occupano di compiti diversi:
\begin{description}
\item[\textsc{reseq\_stampa}]: modulo che effettua la creazione e la scrittura dei file \emph{wiggle}, \acronimo{bam} e \emph{gnuplot} e la stampa su terminale di messaggi riguardanti l'avanzamento del programma;

\item[\textsc{reseq\_utility}]: modulo che contiene alcune funzioni di utilità che vengono usate durante l'analisi dei dati.
\end{description}

\subsection{Script \emph{Gnuplot}}
Per la stampa dei grafici riguardanti la lunghezza degli inserti è stato creato uno script, chiamato \script{plot}, che viene letto ed eseguito da \emph{Gnuplot}; esso produce alcuni file \acronimo{PDF} che riportano sull'asse delle ascisse il numero progressivo degli inserti trovati, mentre sull'asse delle ordinate la lunghezza, in $bp$, degli stessi inserti.

\subsection{Script \emph{Bash}}
\label{sec:bash}
Affinché l'esecuzione dell'intero processo di elaborazione dei dati e produzione dei risultati sia semplice da eseguire ho realizzato un script \emph{bash} che rende tale procedura automatica.
Tale script si occupa di eseguire in sequenza tutte le operazioni necessarie:

\begin{description}
\item[\textsc{Unione file pass}]: effettua il merge dei file \acronimo{pass} ricevuti; crea una versione ordinata per nome e una per posizione di partenza della read sul genoma;
\item[\textsc{Avvio script Python}]: richiama lo script \script{reseq\_new} che effettua l'analisi dei dati;
\item[\textsc{Conversione file bam}]: converte i file \acronimo{bam} in file \acronimo{sam} per una più agevole lettura ad occhio umano;
\item[\textsc{Ordinamento file bam}]: ordina i file \acronimo{bam} creati durante l'esecuzione del programma per consentire di poter effettuare su di essi ulteriori analisi;
\item[\textsc{Conversione file wiggle}]: converte i file \emph{wiggle} in \acronimo{tdf} in modo da poterli utilizzare con \acronimo{igv};
\item[\textsc{Avvio Gnuplot}]: richiama \emph{Gnuplot} per realizzare i grafici relativi alle \emph{insert length}.
\end{description}



