%\clearpage
\section{Esecuzione}
L'esecuzione del programma è gestita mediante lo script \emph{bash} chiamato \script{reseq}; tale script è stato dotato di diverse opzioni che permettono di eseguire il programma con diverse modalità richiamabili tramite gli appositi flag:

\begin{description}
\item[\texttt{-a arg}]: modalità \modo{all}, esegue in sequenza tutte le operazioni descritte nella sezione \ref{sec:bash} e comprende tutte le altre modalità di esecuzione; permette quindi di eseguire l'intero programma ed ottenere tutti i risultati; l'argomento \texttt{arg} è necessario per eseguire la modalità \modo{bam to sam} e quella \modo{start reseq};
\item[\texttt{-b arg}]: modalità \modo{bam to sam}, effettua la conversione di tutti i file \acronimo{bam} presenti nella cartella passata come argomento \texttt{arg} in file \acronimo{sam}, consentendo la visualizzazione all'utente;
\item[\texttt{-g}]: modalità \modo{gnuplot}, richiama \emph{gnuplot} per disegnare i grafici sulla lunghezza degli inserti;
\item[\texttt{-m}]: modalità \modo{merge pass sort}, esegue l'unione dei file \acronimo{pass} forniti, crea una versione ordinata per nome e una per posizione;
\item[\texttt{-s arg}]: modalità \modo{start reseq}, avvia lo script \emph{Python} che esegue l'elaborazione dei dati utilizzando i file \acronimo{bam} di partenza presenti nella cartella passata tramite argomento \texttt{arg};
\item[\texttt{-t}]: modalità \modo{sort bam}, esegue l'ordinamento dei file \acronimo{bam} creati in precedenza e situati nella cartella \texttt{risultati} e crea il relativo indice;
\item[\texttt{-w}]: modalità \modo{wig to tdf}, converte i file \emph{wiggle} creati in file \acronimo{tdf}, pronti per essere visualizzati su \acronimo{igv}.
\end{description}

Affinché l'esecuzione del programma non provochi errori è necessario che la struttura di file sia coerente con quella usata nell'ambiente di sviluppo; insieme alla presente relazione viene fornito un archivio contenente tutti i file sviluppati necessari al funzionamento, non vengono però inclusi quelli relativi ai file \acronimo{pass} di riferimento e al genoma di \emph{A. laidlawii} a causa della loro dimensioni.

Una volta estratto l'archivio in una destinazione a piacere è necessario quindi copiare i file richiesti nelle cartelle corrette; qui di seguito viene descritta la loro struttura nell'ambiente di sviluppo:\\*

%disegna la struttura di cartelle presente nell'archivio consegnato
\tikzstyle{every node}=[thick,anchor=west, rounded corners, font={\scriptsize\ttfamily}, inner sep=2.5pt]
\tikzstyle{selected}=[draw=blue,fill=blue!10]
\tikzstyle{root}=[selected, fill=blue!30]
\tikzstyle{optional}=[dashed, draw=blue, fill=blue!5]

\begin{tikzpicture}[%
    scale=.7,
    grow via three points={one child at (0.5,-0.65) and
    two children at (0.5,-0.65) and (0.5,-1.1)},
    edge from parent path={(\tikzparentnode.south) |- (\tikzchildnode.west)}]
  \node [root] {bioinfo}
    child { node [selected] {pass\_bam}
      child { node {pass\_reads1.bam}}
	  child { node {pass\_reads2.bam}}
    }       
    child { node at (0,-1.5) [selected] {reference\_A\_laidlawii}
      child { node {reference\_A\_laidlawii.fasta}}
      child { node {reference\_A\_laidlawii.fasta.fai}}
    }
    child { node at (0,-3) [optional] {risultati}
      child { node {\dots}}
    }
    child { node at (0,-4) {igv\_session.xml}}
    child { node at (0,-4.2) {plot.gnuplot}}
    child { node at (0,-4.4) {reseq.sh}}
    child { node at (0,-4.6) {reseq\_new.py}}
    child { node at (0,-4.8) {reseq\_stampa.py}}
    child { node at (0,-5) {reseq\_utility.py}};
\end{tikzpicture}

Le cartelle evidenziate in blu chiaro sono obbligatorie mentre quella tratteggiata verrà creata dal programma in fase di esecuzione e, come si evince dal nome, vi verranno salvati i risultati ottenuti man mano.

È presente inoltre il file \script{igv\_session.xml} che rappresenta una sessione di visualizzazione su \acronimo{igv}: è stata creata manualmente e permette, una volta aperto il genome viewer di importare le tracce create dal programma.
La cartella radice può avere un nome a discrezione dell'utente e non influenza il programma in nessun modo.

Devono essere inoltre installati sul sistema e presenti nella variabile di sistema \comando{path} i seguenti software e librerie:
\begin{itemize}
\item Samtools 1.2;
\item Python 3.4;
\item Pysam 0.8.1;
\item \acronimo{igv} 2.3.40;
\item IGVtools 2.3.40;
\item Gnuplot 4.6;
\item Bash 4.3.33;
\end{itemize}

